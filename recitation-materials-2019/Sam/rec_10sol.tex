\documentclass[11pt]{article}
\usepackage[hmargin=35pt,vmargin=35pt]{geometry}
\usepackage{graphicx}
\usepackage{amsfonts}
\usepackage{amsmath}
\usepackage{enumerate}
\pagenumbering{gobble} 
\newcommand{\diff}{\,\mathrm{d}}
\renewcommand*{\vec}[1]{\mathbf{#1}}

\title{18.06 - Recitation 10 - Solutions}
\author{Sam Turton}
\date{May 7, 2019}                                      
\begin{document}
\maketitle

\noindent \textbf{Problem 1.}\\
Consider the matrix $A = \begin{pmatrix} x & 1 & 1 \\ 1 & 1 & 1 \\ 1 & 1 & 1 \end{pmatrix}$ with parameter $x\in\mathbb{R}$:
\begin{enumerate}
\item Specify all numbers $x$, if any, for which $A$ is positive definite. (Explain briefly.)
\item Specify all numbers $x$, if any, for which $e^A$ is positive definite. (Explain briefly.)
\item Find an $x$, if any, for which $4I - A$ is positive definite. (Explain briefly.)
\end{enumerate}

\

\noindent \textbf{Solution}\\
\begin{enumerate}
\item Note that the second and third columns of $A$ are identical. This means that the rank of $A$ is less than 3 and so $\det A =0$. This means that $A$ will always have at least one zero eigenvalue and so it cannot be positive  definite for any $x$
\item Recall that if $A$ has an eigenvalue $\lambda$, then $e^A$ has an eigenvalue $e^{\lambda}$. Since $A$ is symmetric, $\lambda \in \mathbb{R}$, and so $e^{\lambda} > 0$ for every eigenvalue. Hence $e^A$ is positive definite.
\item The upper left determinants are $4-x, 11 - 3x$ and $24 - 8x$. So provided $3>x$, then $4I - A$ will be positive definite.
\end{enumerate}

\

\noindent \textbf{Problem 2.}\\
True or false? Justify your answer either way.
\begin{enumerate}
\item If $A$ and $B$ are invertible, then so is $(A+B)/2$.
\item If $A$ and $B$ are Markov, then so is $(A+B)/2$.
\item If $A$ and $B$ are positive definite, then so is $(A+B)/2$.
\item If $A$ and $B$ are diagonalizable, then so is $(A+B)/2$.
\item If $A$ and $B$ are rank 1, then so is $(A+B)/2$.
\end{enumerate}

\

\noindent \textbf{Solution}\\
\begin{enumerate}
\item This is false. For example $A=I$ and $B=-I$ are both invertible, but $(A+B)/2 = 0$ which is not invertible.
\item This is true. If and $A$ and $B$ are Markov, then all of their entries are nonnegative. So certainly all the entries of $(A+B)/2$ are nonnegative. Furthermore, The columns of $A+B$ are the sums of the columns of $A$ and the columns of $B$. If each of their columns sum to 1, then the columns of $(A+B)/2$ will also sum to 1. So $(A+B)/2$ is also Markov.
\item This is true. If $A$ and $B$ are positive definite, then $x^TAx > 0$ and $x^T B x >0$, for all $x\neq 0 $. So 
$$x^T\left(\frac{A+B}{2}\right)x = \frac{1}{2} (x^TAx + x^TBx) >0.$$
So $(A+B)/2$ is positive definite.
\item This is false. $A = \begin{pmatrix} 2 & 0 \\ 0 & 0 \end{pmatrix}$ and $B = \begin{pmatrix} 0 & 2 \\ 0 & 2 \end{pmatrix}$ are both diagonalizable, but 
$$(A+B)/2 = \begin{pmatrix} 1 & 1 \\ 0 & 1 \end{pmatrix}$$
is not diagonalizable.
\item This is false. $A = \begin{pmatrix} 2 & 0 \\ 2 & 0 \end{pmatrix}$ and $B = \begin{pmatrix} 0 & 2 \\ 0 & -2 \end{pmatrix}$ are both rank 1, but 
$$(A+B)/2 = \begin{pmatrix} 1 & 1 \\ 1 & -1 \end{pmatrix},$$
which is rank 2. 
\end{enumerate}

\

\noindent \textbf{Problem 3.}\\
We are told that $A$ is a symmetric Markov matrix. It has an eigenvalue $y$, where $-1<y<1$. 
\begin{enumerate}
\item Find the matrix $A$ in terms of $y$. 
\item Find the eigenvectors of $A$.
\item What is $\lim_{n\to\infty} A^n$ in its simplest form?
\end{enumerate}

\

\noindent \textbf{Solution}\\
\begin{enumerate}
\item Since $A$ is symmetric, we know that 
$$A = \begin{pmatrix} a & b \\ b & d \end{pmatrix}.$$
Since the columns of $A$ must sum to 1, we know that $ a+b = b+d = 1$, from which we can deduce $a=d$. We can then find the characteristic equation of this matrix:
$$\det (A-\lambda I) = (a-\lambda)^2 - b^2 = 0$$
and so $(a-\lambda) = \pm b$. We know that $a+b = 1$, and so $a-b = y$. Therefore
$$A = \begin{pmatrix} (1+y)/2 & (1-y)/2 \\ (1-y)/2 & (1+y)/2 \end{pmatrix}.$$
\item The eigenvector $\lambda_1 = 1$ is $v_1 = \begin{pmatrix} 1 \\ 1 \end{pmatrix}$. The eigenvector for $\lambda_2 = y$ is $v_2 = \begin{pmatrix} 1 \\ -1 \end{pmatrix}$
\item $A$ is diagonalizable, i.e. we can write $A = X\Lambda X^{-1}$, where
$$ X = \begin{pmatrix} 1 & 1 \\ 1 & -1 \end{pmatrix}, \;\; \Lambda = \begin{pmatrix} 1 & 0 \\ 0 & y \end{pmatrix}, \;\; X^{-1} = \frac{1}{2} \begin{pmatrix} 1 & 1 \\ 1 & -1 \end{pmatrix}$$
Then $A^n = X\Lambda^n X^{-1}$. Now $\Lambda^n \to \begin{pmatrix} 1 & 0 \\ 0 & 0 \end{pmatrix}$, and so
\begin{align*}
A^n &\to  \frac{1}{2}\begin{pmatrix} 1 & 1 \\ 1 & -1 \end{pmatrix}\begin{pmatrix} 1 & 0 \\ 0 & 0 \end{pmatrix} \begin{pmatrix} 1 & 1 \\ 1 & -1 \end{pmatrix}\\
&=  \frac{1}{2}\begin{pmatrix} 1 & 1 \\ 1 & 1 \end{pmatrix}
\end{align*}
\end{enumerate}

\

\noindent \textbf{Problem 4.}\\
\begin{enumerate}
\item If $A$ is symmetric then which of these four matrices are necessarily positive definite
$$A^3, \;\; (A^2+I)^{-1}, \;\; A+I, \;\; e^A.$$
\item Suppose $C$ is positive definite and that $A$ has independent columns. Show that $x^TA^TCAx > 0$ for all $x\neq 0$. Hence $S = A^TCA$ is positive definite. 
\end{enumerate}

\

\noindent \textbf{Solution}\\
\begin{enumerate}
\item $(A^2+I)^{-1}$ and $e^A$ are always positive definite. This is because if $A$ has eigenvalues $\lambda_i\in\mathbb{R}$, then $(A^2+I)^{-1}$ has eigenvalues $\frac{1}{\lambda_i^2+1}>0$ and $e^A$ has eigenvalues $e^{\lambda_i}>0$. $A^3$ has eigenvalues $\lambda_i^3$ which need not be positive, and $A+I$ has eigenvalues $\lambda_i+1$ which also need not be positive. 
\item Let $y = Ax$. Then since $C$ is positive definite, we know that $y^T C y = x^TA^TCAx > 0$ whenever $y=Ax \neq 0$. However, $Ax = 0$ only has $x=0$ as a solution since $A$ has full column rank (independent columns). Hence $y^T C y = x^TA^TCAx > 0$ for all $x\neq 0$ and so  $S = A^TCA$ is positive definite. 
\end{enumerate}
\end{document}  