\documentclass[10pt]{amsart} 


\usepackage{amsmath, amssymb, mathrsfs} 

\usepackage[mathscr]{euscript} 
 
\newlength{\mylength}
\setlength{\mylength}{0.25cm}

\usepackage{enumitem}
\setlist{listparindent=\parindent, itemsep=0cm, parsep=\mylength, topsep=0cm}

\usepackage[final]{todonotes}
\usepackage[final]{showkeys} 

\usepackage[breaklinks=true]{hyperref} 
\usepackage{comment} 

\usepackage{url}

\usepackage{tikz-cd}

\usepackage{amsthm}

\makeatletter
\renewenvironment{proof}[1][\proofname]{\par
	\pushQED{\qed}%
	\normalfont \topsep6\p@\@plus6\p@\relax
	\noindent\emph{#1.} 
	\ignorespaces
}{%
\popQED\endtrivlist\@endpefalse
}
\makeatother

\newtheoremstyle{mythm}% name of the style to be used
{\mylength}% measure of space to leave above the theorem. E.g.: 3pt
{0pt}% measure of space to leave below the theorem. E.g.: 3pt
{\itshape}% name of font to use in the body of the theorem
{0pt}% measure of space to indent
{\bfseries}% name of head font
{.\ }% punctuation between head and body
{ }% space after theorem head; " " = normal interword space
{\thmname{#1}\thmnumber{ #2}\thmnote{ (#3)}}

\newtheoremstyle{myrmk}% name of the style to be used
{\mylength}% measure of space to leave above the theorem. E.g.: 3pt
{0pt}% measure of space to leave below the theorem. E.g.: 3pt
{}% name of font to use in the body of the theorem
{0pt}% measure of space to indent
{\itshape}% name of head font
{.\ }% punctuation between head and body
{ }% space after theorem head; " " = normal interword space
{\thmname{#1}\thmnumber{ #2}\thmnote{ (#3)}}

\theoremstyle{mythm} 
%\newtheorem{thm}[subsubsection]{Theorem}
%\newtheorem*{claim}{Claim}
%\newtheorem*{thm}{Theorem} 
\newtheorem{thm}{Theorem}
\newtheorem{lem}[thm]{Lemma} 
\newtheorem{cor}[thm]{Corollary}
\newtheorem{claim}[thm]{Claim}
\newtheorem{prop}[thm]{Proposition}
%\newtheorem*{mthm}{Main Theorem}

%\newtheorem{prop}[subsubsection]{Proposition} 
%\newtheorem*{prop}{Proposition} 
%\newtheorem*{lem}{Lemma}
%\newtheorem*{klem}{Key Lemma}
%\newtheorem*{cor}{Corollary}

\theoremstyle{definition}
%\newtheorem{defn}[subsubsection]{Definition}
\newtheorem*{defn}{Definition} 
\newtheorem{prob}[thm]{Problem}
%\newtheorem{que}[subsubsection]{Question}

\theoremstyle{myrmk} 
%\newtheorem{rmk}[subsubsection]{Remark}
\newtheorem*{rmk}{Remark}
%\newtheorem{note}[subsubsection]{Note} 
\newtheorem*{ex}{Example}

\newcommand{\nc}{\newcommand} 
\nc{\on}{\operatorname}
\nc{\rnc}{\renewcommand} 

\rnc{\setminus}{\smallsetminus} 

\nc{\wt}{\widetilde}
\nc{\wh}{\widehat} 
\nc{\ol}{\overline} 

\nc{\Frob}{\on{Frob}}
\nc{\Gal}{\on{Gal}}

\nc{\BN}{\mathbb{N}}
\nc{\BZ}{\mathbb{Z}}
\nc{\BQ}{\mathbb{Q}}
\nc{\BR}{\mathbb{R}}
\nc{\BC}{\mathbb{C}}

\nc{\id}{\on{id}}
\nc{\Id}{\on{Id}}
\nc{\Tr}{\on{Tr}}

\nc{\la}{\langle}
\nc{\ra}{\rangle} 
\nc{\lV}{\lVert}
\nc{\rV}{\rVert}
\nc{\mb}{\mathbf}
\nc{\mf}{\mathfrak}
%\nc{\cur}{\mathscr}
\nc{\mc}{\mathscr}

\nc{\ira}{\hookrightarrow}
\nc{\hra}{\hookrightarrow}
\nc{\sra}{\twoheadrightarrow} 

\rnc{\Re}{\on{Re}}

\nc{\coker}{\on{coker}}
\nc{\End}{\on{End}}
\rnc{\Im}{\on{Im}}
%\rnc{\Re}{\on{Re}}

\nc{\Hom}{\on{Hom}}

\DeclareMathOperator*{\argmin}{arg\,min}
\DeclareMathOperator*{\argmax}{arg\,max}

\usepackage{marginnote}
\nc{\acts}{\curvearrowright}

\nc{\Mat}{\on{Mat}}

\newenvironment{cd}{\begin{equation*}\begin{tikzcd}}{\end{tikzcd}\end{equation*}\ignorespacesafterend}

\nc{\pfrac}[2]{\frac{\partial #1}{\partial #2}}
\nc{\e}[1]{\begin{align*} #1 \end{align*}}

\usepackage[margin=1in]{geometry}

\makeatletter
\def\blfootnote{\gdef\@thefnmark{}\@footnotetext}
\makeatother

%\renewcommand*{\arraystretch}{1.4}

\setlength{\parskip}{0.25cm}

\newenvironment{myproof}{\color{blue}\begin{proof}}{\end{proof}} 



\usepackage{fancyhdr}
\pagestyle{fancy} 
\fancyhead[L]{James Tao}
\fancyhead[C]{18.06 -- Week 7 Recitation}
\fancyhead[R]{Mar.\ 31, 2020}
\fancyfoot[C]{}

\newcounter{part-count}
\setcounter{part-count}{0}

\newenvironment{me}[1]{\begin{enumerate}[#1]\setcounter{enumi}{\value{part-count}}}{\setcounter{part-count}{\value{enumi}}\end{enumerate}}


\begin{document}
	\thispagestyle{fancy}
	
	\noindent Determine whether or not these objects exist. If so, write down an example. If not, explain why not. 
	
	\begin{me}{itemsep = 0.2cm}
		\item A $2 \times 4$ matrix $A$ such that $\on{null}(A) = \left\{ \begin{pmatrix}
		x \\ 0 \\ 0 \\ 0 
		\end{pmatrix} \text{ for all } x \in \BR\right\}$. 
		\item An invertible matrix of the following form: 
		\[
		\begin{pmatrix}
			? & ? & ? & ? \\ 
			? & ? & ? & ? \\
			? & 0 & 0 & 0 \\
			? & 0 & 0 & 0
		\end{pmatrix}
		\]
		\item Let $A = \begin{pmatrix}
		1 & 2 & 3 \\ 4 & 5 & 6 \\ 7 & 8 & 9 
		\end{pmatrix}$. A matrix $B$ such that $\on{col}(B) = \on{row}(A)$ and $\on{null}(B) = \on{null}(A^\top)$. 
		\item A matrix $A = \begin{pmatrix}
			a_{11} & a_{12} \\ a_{21} & a_{22} 
		\end{pmatrix}$ such that the system 
		\e{
			a_{11}x + a_{12}y &= 1 \\
			a_{21}x + a_{22}y &= 2
		} 
		has no solution, the system 
		\e{
			a_{11}x + a_{12}y &= 1 \\
			a_{21}x + a_{22}y &= 1
		} 
		has exactly one solution, and the system 
		\e{
			a_{11}x + a_{12}y &= 1 \\
			a_{21}x + a_{22}y &= 0
		} 
		has infinitely many solutions. 
		\item Two subspaces $V_1, V_2 \subseteq \BR^3$ such that $\dim(V_1) = \dim(V_2) = 2$ and $V_1 \cap V_2 = \{0\}$. 
		\item Let $A = \begin{pmatrix}
		1 & 1 & 1
		\end{pmatrix}$. A basis $\{v_1, v_2, v_3\} \subset \BR^3$ such that $v_1 \in \on{null}(A)$ and $v_2, v_3 \in \on{row}(A)$. 
		\item Let $A = \begin{pmatrix}
		1 & 1 & 1
		\end{pmatrix}$. A basis $\{v_1, v_2, v_3\} \subset \BR^3$ such that $v_1, v_2 \in \on{null}(A)$ and $v_3 \in \on{row}(A)$.  
		\item An $n \times n$ matrix $P$ such that $P^2 = P$, $\on{rank}(P) = n$, and $P \neq \on{Id}_{n \times n}$. 
		\item An $n \times n$ matrix $A$ such that all singular values of $A$ are equal to 1, and $A \neq \on{Id}_{n \times n}$. 
		\item Two subspaces $V_1, V_2 \subseteq \BR^3$ such that $V_1 \cap V_2 = \{0\}$ and $V_1^\perp \cap V_2^\perp = \{0\}$. 
		\item A $3 \times 5$ matrix $A$ such that $\dim(\on{null}(A)) + \dim(\on{null}(A^\top)) = 5$. 
		\item Matrices $A$ and $B$ such that $\on{pinv}(A) = \on{pinv}(B)$ and $A \neq B$. 
		\item Matrices $A$ and $B$ such that $A^\top A = B^\top B$ and $A \neq B$. 
		\item A square matrix $A$ such that $A^\top A + AA^\top$ is noninvertible. 
		\item An $m \times n$ matrix $A$ and a nonzero vector $v \in \on{row}(A)$ such that $Av \in \on{null}(A^\top)$. 
	\end{me}
	
	\newpage
	
	\section*{Solutions} 
	
	\noindent DNE stands for `does not exist.' 
	
	\begin{enumerate}
		\item DNE. Combine the relation $\on{rank}(A) + \dim(\on{null}(A)) = 4$ with $\on{rank}(A) \le 2$ and $\dim(\on{null}(A)) = 1$ to get a contradiction. 
		\item DNE. The last three columns of $A$ provide three vectors in a 2-dimensional space, so they can't be linearly independent. 
		\item Take $B = A^\top$. 
		\item DNE. If $\on{null}(A) > 0$, then every system of the form $A\mb{x} = \mb{b}$ has either zero or infinitely many solutions. If $\on{null}(A) = 0$, then every such system has either zero or one solution. 
		\item DNE. Suppose such $V_1, V_2$ did exist. Take a basis $\{v_1, v_2\}$ for $V_1$ and a basis $\{w_1, w_2\}$ for $V_2$. Since $\{v_1, v_2, w_1, w_2\}$ is a list of four vectors in $\BR^3$, they must satisfy some nontrivial linear relation. If this linear relation involves both $\{v_1, v_2\}$ and $\{w_1, w_2\}$, then we can rearrange it to look as follows: 
		\[
			c_1v_1 + c_2v_2 = d_1w_1 + d_2w_2
		\]
		where $c_1$ and $c_2$ are not both zero, and $d_1$ and $d_2$ are not both zero. This contradicts $V_1 \cap V_2 = \{0\}$ because the LHS lies in $V_1 \setminus \{0\}$ while the RHS lies in $V_2 \setminus \{0\}$. 
		
		The only remaining possibilities are that the linear relation only involves $\{v_1, v_2\}$ or only involves $\{w_1, w_2\}$. Neither is possible, because these are bases. 
		\item DNE. Since $\{v_1, v_2, v_3\}$ is a basis, $\{v_2, v_3\}$ must be linearly independent. But $\on{row}(A)$ is one-dimensional, so it doesn't contain a pair of linearly independent vectors. 
		\item One may take  
		\[
			v_1 = \begin{pmatrix}
			1 \\ -1 \\ 0 
			\end{pmatrix}, \qquad v_2 = \begin{pmatrix}
			1 \\ 0 \\ -1
			\end{pmatrix}, \qquad v_3 = \begin{pmatrix}
			1 \\ 1 \\ 1
			\end{pmatrix}. 
		\]
		\item DNE. Since $\on{rank}(P) = n$, we know that $\on{col}(P) = \BR^n$. Therefore any vector in $\BR^n$ can be written as $Px$ for some $x \in \BR^n$. 
		
		Next, we claim that $Py = y$ for all $y \in \BR^n$. To see this, write $y = Px$ for some $x$, after which the equation becomes $P^2x = Px$ which is part of the hypothesis. 
		
		This implies that $P = \on{Id}_{n \times n}$. 
		\item Take $A$ to be any square orthogonal matrix which is not the identity. 
		\item Take 
		\[
			V_1 = \on{span}\left( \begin{pmatrix}
			1 \\ 0 \\ 0
			\end{pmatrix} \right) \qquad \text{ and } \qquad V_2 = \on{span}\left( \begin{pmatrix}
			0 \\ 1 \\ 0
			\end{pmatrix}, \begin{pmatrix}
			0 \\ 0 \\ 1 
			\end{pmatrix}\right)
		\]
		Then $V_1^\perp = V_2$ and $V_1 \cap V_2 = 0$, from which it also follows that $V_1^\perp \cap V_2^\perp = \{0\}$. 
		\item DNE. If $r$ is the rank of $A$, we have $\dim(\on{null}(A)) = 5-r$ and $\dim(\on{null}(A^\top)) = 3-r$, so the desired equation implies $8-2r=5$. This is impossible since the RHS can't even. 
		\item DNE. The construction of the pseudoinverse mentioned in class (based on the SVD) implies that $\on{pinv}(\on{pinv}(A)) = A$ for any matrix $A$. Therefore, if $\on{pinv}(A) = \on{pinv}(B)$, applying $\on{pinv}(-)$ to both sides implies that $A = B$. 
		\item Take $A$ to be any nonidentity orthogonal matrix, and take $B$ to be the identity. 
		\item Take $A = \begin{pmatrix}
		1 & 0 \\ 0 & 0 
		\end{pmatrix}$. 
		\item DNE. Since $Av \in \on{null}(A^\top)$ by hypothesis and $Av \in \on{col}(A)$ by definition, we conclude that $Av \in \on{null}(A^\top) \cap \on{col}(A) = \{0\}$, so $Av = 0$. Therefore $v \in \on{null}(A)$ by definition. Thus, $v \in \on{row}(A) \cap \on{null}(A) = \{0\}$, which contradicts that $v$ is nonzero. 
	\end{enumerate} 
	
\end{document} 