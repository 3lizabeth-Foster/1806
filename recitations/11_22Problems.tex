\documentclass{article}
\usepackage{amsmath, amsthm, amssymb, amsfonts, dsfont, fancyhdr, graphicx, color, tabularx, enumitem}
\usepackage{geometry}


\theoremstyle{definition}
\newtheorem{prob}{}
\renewcommand{\qedsymbol}{}
\renewcommand*{\proofname}{Solution}
\newcommand{\MSB}[1]{\textcolor{blue}{[MSB: #1]}}
\def\tr{\text{tr}}

\pagestyle{fancy} \fancyhf{} \lhead{\textsc{18.06}} \rhead{11/22/22} 

\begin{document}


\section*{Practice Problems}
\begin{prob}
	Say $A$ is a $3 \times 3$ real matrix with eigenvalues $\lambda_1=-1, \lambda_2=-3+4i, \lambda_3=-3-4i$, with corresponding eigenvectors $x_1, x_2, x_3$. 
	\begin{itemize}
		\item[a)] What are the trace and determinant of $2A$?
		\item[b)] Two eigenvectors of $A$ are $x_1=\begin{pmatrix}
			1\\0\\0
		\end{pmatrix}$
	and 
	$x_2=\begin{pmatrix}
		0\\1\\i
	\end{pmatrix}$.
What is $x_3$?
	\end{itemize}
\end{prob}


\begin{prob}
	Using the same matrix $A$ as in 1, which of the following has unbounded magnitude (i.e. magnitude blowing up) as $n \to \infty$ or $t \to \infty$? Assume $y$ is chosen at random.
	\begin{itemize}
		\item[a)] $A^n y$ as $n \to \infty$
		\item[b)] $A^{-n}y$ as $n \to \infty$
		\item[c)] The solution of $\frac{dx}{dt}=Ax$ as $t \to \infty$ for the initial condition $x(0)=y$.
		\item[d)] The solution of $\frac{dx}{dt}=-Ax$ as $t \to \infty$ for the initial condition $x(0)=y$.
	\end{itemize}
\end{prob}



\begin{prob}
	Using the same matrix $A$ as in 1, write down the exact solution $x(t)$ to $\frac{dx}{dt}= Ax$ for the initial condition $x(0)= \begin{pmatrix}
		1\\2\\0
	\end{pmatrix}$.
\end{prob}


\begin{prob}
	Use the series formula for $e^{At}$ to show that
	\[\frac{d}{dt}e^{At}= A e^{At}.\]
	Use this to conclude that $x(t)= e^{At}x(0)$ satisfies $\frac{dx}{dt}= Ax$.
\end{prob}

\end{document}