\documentclass[10pt]{amsart} 


\usepackage{amsmath, amssymb, mathrsfs} 

\usepackage[mathscr]{euscript} 
 
\newlength{\mylength}
\setlength{\mylength}{0.25cm}

\usepackage{enumitem}
\setlist{listparindent=\parindent, itemsep=0cm, parsep=\mylength, topsep=0cm}

\usepackage[final]{todonotes}
\usepackage[final]{showkeys} 

\usepackage[breaklinks=true]{hyperref} 
\usepackage{comment} 

\usepackage{url}

\usepackage{tikz-cd}

\usepackage{amsthm}

\makeatletter
\renewenvironment{proof}[1][\proofname]{\par
	\pushQED{\qed}%
	\normalfont \topsep6\p@\@plus6\p@\relax
	\noindent\emph{#1.} 
	\ignorespaces
}{%
\popQED\endtrivlist\@endpefalse
}
\makeatother

\newtheoremstyle{mythm}% name of the style to be used
{\mylength}% measure of space to leave above the theorem. E.g.: 3pt
{0pt}% measure of space to leave below the theorem. E.g.: 3pt
{\itshape}% name of font to use in the body of the theorem
{0pt}% measure of space to indent
{\bfseries}% name of head font
{.\ }% punctuation between head and body
{ }% space after theorem head; " " = normal interword space
{\thmname{#1}\thmnumber{ #2}\thmnote{ (#3)}}

\newtheoremstyle{myrmk}% name of the style to be used
{\mylength}% measure of space to leave above the theorem. E.g.: 3pt
{0pt}% measure of space to leave below the theorem. E.g.: 3pt
{}% name of font to use in the body of the theorem
{0pt}% measure of space to indent
{\itshape}% name of head font
{.\ }% punctuation between head and body
{ }% space after theorem head; " " = normal interword space
{\thmname{#1}\thmnumber{ #2}\thmnote{ (#3)}}

\theoremstyle{mythm} 
%\newtheorem{thm}[subsubsection]{Theorem}
%\newtheorem*{claim}{Claim}
%\newtheorem*{thm}{Theorem} 
\newtheorem{thm}{Theorem}
\newtheorem{lem}[thm]{Lemma} 
\newtheorem{cor}[thm]{Corollary}
\newtheorem{claim}[thm]{Claim}
\newtheorem{prop}[thm]{Proposition}
%\newtheorem*{mthm}{Main Theorem}

%\newtheorem{prop}[subsubsection]{Proposition} 
%\newtheorem*{prop}{Proposition} 
%\newtheorem*{lem}{Lemma}
%\newtheorem*{klem}{Key Lemma}
%\newtheorem*{cor}{Corollary}

\theoremstyle{definition}
%\newtheorem{defn}[subsubsection]{Definition}
\newtheorem*{defn}{Definition} 
\newtheorem{prob}[thm]{Problem}
%\newtheorem{que}[subsubsection]{Question}

\theoremstyle{myrmk} 
%\newtheorem{rmk}[subsubsection]{Remark}
\newtheorem*{rmk}{Remark}
%\newtheorem{note}[subsubsection]{Note} 
\newtheorem*{ex}{Example}

\newcommand{\nc}{\newcommand} 
\nc{\on}{\operatorname}
\nc{\rnc}{\renewcommand} 

\rnc{\setminus}{\smallsetminus} 

\nc{\wt}{\widetilde}
\nc{\wh}{\widehat} 
\nc{\ol}{\overline} 

\nc{\Frob}{\on{Frob}}
\nc{\Gal}{\on{Gal}}

\nc{\BN}{\mathbb{N}}
\nc{\BZ}{\mathbb{Z}}
\nc{\BQ}{\mathbb{Q}}
\nc{\BR}{\mathbb{R}}
\nc{\BC}{\mathbb{C}}

\nc{\id}{\on{id}}
\nc{\Id}{\on{Id}}
\nc{\Tr}{\on{Tr}}

\nc{\la}{\langle}
\nc{\ra}{\rangle} 
\nc{\lV}{\lVert}
\nc{\rV}{\rVert}
\nc{\mb}{\mathbf}
\nc{\mf}{\mathfrak}
%\nc{\cur}{\mathscr}
\nc{\mc}{\mathscr}

\nc{\ira}{\hookrightarrow}
\nc{\hra}{\hookrightarrow}
\nc{\sra}{\twoheadrightarrow} 

\rnc{\Re}{\on{Re}}

\nc{\coker}{\on{coker}}
\nc{\End}{\on{End}}
\rnc{\Im}{\on{Im}}
%\rnc{\Re}{\on{Re}}

\nc{\Hom}{\on{Hom}}

\DeclareMathOperator*{\argmin}{arg\,min}
\DeclareMathOperator*{\argmax}{arg\,max}

\usepackage{marginnote}
\nc{\acts}{\curvearrowright}

\nc{\Mat}{\on{Mat}}

\newenvironment{cd}{\begin{equation*}\begin{tikzcd}}{\end{tikzcd}\end{equation*}\ignorespacesafterend}

\nc{\pfrac}[2]{\frac{\partial #1}{\partial #2}}
\nc{\e}[1]{\begin{align*} #1 \end{align*}}

\usepackage[margin=1in]{geometry}

\makeatletter
\def\blfootnote{\gdef\@thefnmark{}\@footnotetext}
\makeatother

%\renewcommand*{\arraystretch}{1.4}

\setlength{\parskip}{0.25cm}

\newenvironment{myproof}{\color{blue}\begin{proof}}{\end{proof}} 


\usepackage{fancyhdr}
\pagestyle{fancy} 
\fancyhead[L]{James Tao}
\fancyhead[C]{18.06 -- Proof of existence of SVD}
\fancyhead[R]{Mar.\ 6, 2020}
\fancyfoot[C]{}


\begin{document}
	\thispagestyle{fancy}
	
	{\color{blue} The material in this document will not be used or required in the class. } 
	
	Let $A$ be an $n \times m$ matrix. We prove that $A$ has a (full) SVD $A = U\Sigma V^\top$. 
	
	First, we reduce our problem to finding an orthonormal basis of $\BR^m$ whose property of orthogonality is preserved under the transformation defined by $A$. 
	
	\begin{lem}
		Assume there exists an orthonormal basis $v_1, \ldots, v_m \in \BR^m$ such that the vectors $Av_i$ are pairwise orthogonal, meaning that $(Av_i) \cdot (Av_j) = 0$ for all $i \neq j$. Then $A$ has an SVD. 
	\end{lem}
	\begin{proof}
		Some of the $Av_i$'s may be zero. Reorder the $v_i$'s so that those ones come last. Hence, for some $r$, we may assume that 
		\begin{itemize}
			\item $Av_1, \ldots, Av_r$ are nonzero and of weakly decreasing length. 
			\item $Av_{r+1}, \ldots, Av_m$ are zero. 
		\end{itemize}
		Let $V$ be the $m \times m$ orthogonal matrix built from $v_1, \ldots, v_m$. Let $\Sigma$ be the $n \times m$ diagonal matrix whose first $r$ diagonal entries are $\lVert Av_1\rVert, \ldots, \lVert Av_r\rVert$, and whose remaining entries are zero. Let $U$ be an $n \times n$ orthogonal matrix whose first $r$ columns are given by the orthonormal collection $\frac{1}{\lVert Av_1\rVert} Av_1, \ldots, \frac{1}{\lVert Av_r\rVert} Av_r$, and whose remaining columns are arbitrary. Then we have 
		\[
			Av_i = \lVert Av_i \rVert \, (i\text{-th column of } U)
		\]
		for all $i = 1, \ldots, m$. Therefore $AV = U \Sigma$, so $A = U \Sigma V^\top$, as desired. 
	\end{proof}
	
	Next, we show that this `good' orthonormal basis exists. 
	
	\begin{lem} \label{l2} 
		Let $v \in \BR^m$ be a unit vector which maximizes $\lVert Av \rVert^2$. Then, for any $w \in \BR^m$ such that $w \cdot v = 0$, we have $(Aw) \cdot (Av) = 0$. 
	\end{lem}
	\begin{proof}
		The maximality property of $v$ implies that $t = 0$ is a global maximum of the function 
		\[
			f(t) = \left\lVert A\left( \tfrac{v + tw}{\lVert v + tw \rVert} \right) \right \rVert^2, 
		\]
		because $\frac{v + tw}{\lVert v + tw \rVert}$ is a unit vector. Therefore, $f'(0) = 0$. 
		
		This derivative is computed as follows. First, note that 
		\e{
			f(t) &= \tfrac{1}{\lVert v + tw \rVert^2} \, (v + tw)^\top A^\top A (v + tw) \\
			&= \tfrac{1}{1 + t^2 \lVert w \rVert^2} \, \left( \lVert Av \rVert^2 + 2t\,  (Aw) \cdot (Av) + t^2\, \lVert Aw \rvert^2 \right) 
		}
		where we have used that $\lVert v \rVert = 1$ and $v \cdot w = 0$. By looking at the $t$ coefficient and ignoring higher powers of $t$, we see that $f'(0) = 2\, (Aw) \cdot (Av)$. Therefore $(Aw) \cdot (Av) = 0$, as desired. 		
	\end{proof}
	
	\begin{lem}
		Let $A$ be an $n \times m$ matrix. There exists an orthonormal basis $v_1, \ldots, v_m \in \BR^m$ such that the vectors $Av_i$ are pairwise orthogonal. 
	\end{lem}
	\begin{proof}
		Proceed by induction on $m$. For the base case $m = 1$, just take $v_1$ to be any unit vector. 
		
		Assume $m \ge 2$. Let $v_m$ be a unit vector\footnote{Such a $v_m$ exists because any continuous function on a compact domain attains its maximum. Indeed, a unit vector $v$ varies on the unit sphere in $\BR^m$, which is compact, and the map $v \mapsto \lVert Av \rVert^2$ is continuous.}  which maximizes $\lVert Av_m \rVert^2$. Consider the $(m-1)$-dimensional subspace $W \subset \BR^m$ consisting of all vectors $w$ such that $w \cdot v = 0$. By choosing an orthonormal basis of $W$, we obtain an $m \times (m-1)$ orthogonal matrix $Q$ whose column space is $W$. Applying the inductive hypothesis to the $n \times (m-1)$ matrix $AQ$, we obtain orthonormal vectors $x_1, \ldots, x_{m-1} \in \BR^{m-1}$ such that the $AQx_i$ are pairwise orthogonal. Since $Q$ is an orthogonal matrix, the vectors $Qx_i$ are orthonormal. 
		
		Set $v_i = Qx_i$ for $i = 1, \ldots, m-1$. Then the $v_1, \ldots, v_{m-1}$ are orthonormal, and $Av_1, \ldots, Av_{m-1}$ are pairwise orthogonal. Since $v_1, \ldots, v_{m-1} \in W$ by construction, the set $v_1, \ldots, v_m$ is an orthonormal basis, and Lemma~\ref{l2} implies that $Av_m$ is orthogonal to the $Av_1, \ldots, Av_{m-1}$. Therefore $v_1, \ldots, v_m$ has the desired properties, so the inductive step is proved. 
	\end{proof}
	
\end{document} 