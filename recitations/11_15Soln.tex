\documentclass{article}
\usepackage{amsmath, amsthm, amssymb, amsfonts, dsfont, fancyhdr, graphicx, color, tabularx, enumitem}
\usepackage{geometry}


\theoremstyle{definition}
\newtheorem{prob}{}
\renewcommand{\qedsymbol}{}
\renewcommand*{\proofname}{Solution}
\newcommand{\MSB}[1]{\textcolor{blue}{[MSB: #1]}}
\def\tr{\text{tr}}

\pagestyle{fancy} \fancyhf{} \lhead{\textsc{18.06}} \rhead{11/15/22} 

\begin{document}


\section*{Practice Problems}
\begin{prob}
	Consider the matrix
	\[A= \begin{pmatrix}
		1 & -1 & -1\\-1&1&-1\\-1&-1&1
	\end{pmatrix}.\]
One eigenvalue of $A$ is $-1$. The other eigenvalue is a double root of $\det(A-\lambda I)$. What is the other eigenvalue? (Hint: use the trace, don't do much calculation.)
\end{prob}

\begin{proof}
	We will use the fact that the trace of $A$ is the sum of its eigenvalues. Say the mystery eigenvalue is $\lambda$. Then we have
	\begin{align*}
		3&= -1 + 2 \lambda\\
		4&= 2 \lambda\\
		\lambda&=2.
	\end{align*}
\end{proof}


\begin{prob}
Suppose $M$ is a positive Markov matrix (so one eigenvalue equals $1$, all other eigenvalues have $|\lambda|<1$).  Why is $M^\infty$ a rank-1 matrix?
\end{prob}
\begin{proof}
	We know that the eigenvalues of $M^n$ are the $n$th powers of eigenvalues of $M$. So for $n$ large, $M^n$ has one eigenvalue equal to $1$ and the remaining eigenvalues are extremely small numbers. This means that $M^\infty$ has one eigenvalue equal to 1 and all other eigenvalues equal to zero. The number of nonzero eigenvalues is the rank of the matrix (the number of zero eigenvalues is the dimension of the nullspace).
\end{proof}

\begin{prob}
	$x^T A y = \text{tr}(A B)$ where $B$ is what matrix?   (Hint: use the cyclic property of the trace and recall that the trace of a 1x1 matrix $a$ is $a$).
\end{prob}

\begin{proof}
	We would like to get a trace on the left hand side. The left hand side is a scalar, so is equal to its own trace. That is. $x^T A y=\text{tr}(x^T A y)$. We also know that $\tr(BC)=\tr(CB)$ for any $B, C$. So $\tr(x^T A y)= \tr(Ay x^T)$, and $B=y x^T$.
\end{proof}

\begin{prob}

Suppose $A$ is an $m \times n$ full column-rank matrix with thin SVD $U\Sigma V^T$ (so that $V$ is square/unitary and $U$ is $m \times n$).   By inspection of $A^T A$ in comparison with the diagonalization formula, the eigenvectors of $A^T A$ are \_\_\_ and its eigenvalues are \_\_\_ .
\end{prob}

\begin{proof}
	First, we compute $A^TA:$
	\[A^TA= V \Sigma U^T (U \Sigma V^T)= V \Sigma^2 V^T\]
	since $U$ has orthonormal columns so $U^TU=I$. The diagonalization formula is $B= X \Lambda X^{-1}$, where the columns of $X$ are the eigenvectors of $B$ and $\Lambda$ is diagonal and the diagonal entries are eigenvalues of $B$. Comparing, we see that the columns of $V$ are the eigenvectors of $A^TA$ and the eigenvalues are $\sigma^2$, the squares of the singular values.
\end{proof}

\begin{prob}
	Suppose $A$ is $m \times m$, full rank, and we compute its QR factorization $A=QR$, e.g. by Gram–Schmidt.   Claim: the matrix $B=RQ$ has the same eigenvalues as $A$.  Why?
\end{prob}
\begin{proof}
	We start by relating $B$ to $A$. Since $A=QR$, the matrix $$RQ=Q^{-1} (QR) Q= Q^{-1} A Q.$$ This tells us that $A$ and $RQ$ are similar matrices, so they have the same eigenvalues.
\end{proof}



\end{document}