\documentclass{article}
\usepackage{amsmath, amsthm, amssymb, amsfonts, dsfont, fancyhdr, graphicx, color, tabularx, enumitem}
\usepackage{geometry}


\theoremstyle{definition}
\newtheorem{prob}{}
\renewcommand{\qedsymbol}{}
\renewcommand*{\proofname}{Solution}
\newcommand{\MSB}[1]{\textcolor{blue}{[MSB: #1]}}

\pagestyle{fancy} \fancyhf{} \lhead{\textsc{18.06}} \rhead{11/8/22} 

\begin{document}


\section*{Practice Problems}

\begin{prob}
	The ``tribonacci numbers" are the sequence defined by $T_1=1$, $T_2=1,$ $T_3=2$ and the recurrence
	\[T_n=T_{n-1}+ T_{n-2}+ T_{n-3}.\]
	\begin{itemize}
		\item[a)] Write this 3-term recurrence as a square matrix $R$ acting on the last \_\_\_\_\_ numbers in this sequence. 
		\item[b)] Find a formula for $T_n$ involving only $T_1, T_2, T_3$ and $R$.
		\item[c)] The eigenvalues of this matrix are roughly $-0.42 \pm 0.61 i$ and $1.84$. What does this tell you about the behavior of the Tribonacci numbers for large $n$?
	\end{itemize}

\end{prob}



\begin{prob}
	Consider the matrix 
	\[\begin{pmatrix}
		1 & 1\\ -2 & 4
	\end{pmatrix}\]
from lecture, which has eigenvalues $\lambda_1=2$ and $\lambda_2=3$, and corresponding eigenvectors $x_1=[1 \ 1]$ and $x_2=[1\ 2]$.
\begin{itemize}
	\item[a)] What do we get if we take the vector $x= [3 \ 4]= 2x_1 + x_2$ and multiply 100 times by $A^{-1}$?
	\item[b)] What happens if we take $x$ and multiply many times by $(2A-5I)^{-1}$? Does it converge to a particular vector?
	\item[c)] More generally, if we have an arbitrary matrix $A$ with all eigenvalues distinct, and we multiply a vector $x$ repeatedly by $A^{-1}$, it typically approaches what eigenvector? When might this fail to happen?
\end{itemize}
\end{prob}


\end{document}