\documentclass[10pt]{amsart} 


\usepackage{amsmath, amssymb, mathrsfs} 

\usepackage[mathscr]{euscript} 
 
\newlength{\mylength}
\setlength{\mylength}{0.25cm}

\usepackage{enumitem}
\setlist{listparindent=\parindent, itemsep=0cm, parsep=\mylength, topsep=0cm}

\usepackage[final]{todonotes}
\usepackage[final]{showkeys} 

\usepackage[breaklinks=true]{hyperref} 
\usepackage{comment} 

\usepackage{graphicx}

\usepackage{url}

\usepackage{tikz-cd}

\usepackage{amsthm}

\makeatletter
\renewenvironment{proof}[1][\proofname]{\par
	\pushQED{\qed}%
	\normalfont \topsep6\p@\@plus6\p@\relax
	\noindent\emph{#1.} 
	\ignorespaces
}{%
\popQED\endtrivlist\@endpefalse
}
\makeatother

\newtheoremstyle{mythm}% name of the style to be used
{\mylength}% measure of space to leave above the theorem. E.g.: 3pt
{0pt}% measure of space to leave below the theorem. E.g.: 3pt
{\itshape}% name of font to use in the body of the theorem
{0pt}% measure of space to indent
{\bfseries}% name of head font
{.\ }% punctuation between head and body
{ }% space after theorem head; " " = normal interword space
{\thmname{#1}\thmnumber{ #2}\thmnote{ (#3)}}

\newtheoremstyle{myrmk}% name of the style to be used
{\mylength}% measure of space to leave above the theorem. E.g.: 3pt
{0pt}% measure of space to leave below the theorem. E.g.: 3pt
{}% name of font to use in the body of the theorem
{0pt}% measure of space to indent
{\itshape}% name of head font
{.\ }% punctuation between head and body
{ }% space after theorem head; " " = normal interword space
{\thmname{#1}\thmnumber{ #2}\thmnote{ (#3)}}

\theoremstyle{mythm} 
%\newtheorem{thm}[subsubsection]{Theorem}
%\newtheorem*{claim}{Claim}
%\newtheorem*{thm}{Theorem} 
\newtheorem{thm}{Theorem}
\newtheorem{lem}[thm]{Lemma} 
\newtheorem{cor}[thm]{Corollary}
\newtheorem{claim}[thm]{Claim}
\newtheorem{prop}[thm]{Proposition}
%\newtheorem*{mthm}{Main Theorem}

%\newtheorem{prop}[subsubsection]{Proposition} 
%\newtheorem*{prop}{Proposition} 
%\newtheorem*{lem}{Lemma}
%\newtheorem*{klem}{Key Lemma}
%\newtheorem*{cor}{Corollary}

\theoremstyle{definition}
%\newtheorem{defn}[subsubsection]{Definition}
\newtheorem*{defn}{Definition} 
\newtheorem{prob}[thm]{Problem}
%\newtheorem{que}[subsubsection]{Question}

\theoremstyle{myrmk} 
%\newtheorem{rmk}[subsubsection]{Remark}
\newtheorem*{rmk}{Remark}
%\newtheorem{note}[subsubsection]{Note} 
\newtheorem*{ex}{Example}

\newcommand{\nc}{\newcommand} 
\nc{\on}{\operatorname}
\nc{\rnc}{\renewcommand} 

\rnc{\setminus}{\smallsetminus} 

\nc{\wt}{\widetilde}
\nc{\wh}{\widehat} 
\nc{\ol}{\overline} 

\nc{\Frob}{\on{Frob}}
\nc{\Gal}{\on{Gal}}

\nc{\BN}{\mathbb{N}}
\nc{\BZ}{\mathbb{Z}}
\nc{\BQ}{\mathbb{Q}}
\nc{\BR}{\mathbb{R}}
\nc{\BC}{\mathbb{C}}

\nc{\id}{\on{id}}
\nc{\Id}{\on{Id}}
\nc{\Tr}{\on{Tr}}

\nc{\la}{\langle}
\nc{\ra}{\rangle} 
\nc{\lV}{\lVert}
\nc{\rV}{\rVert}
\nc{\mb}{\mathbf}
\nc{\mf}{\mathfrak}
%\nc{\cur}{\mathscr}
\nc{\mc}{\mathscr}

\nc{\ira}{\hookrightarrow}
\nc{\hra}{\hookrightarrow}
\nc{\sra}{\twoheadrightarrow} 

\rnc{\Re}{\on{Re}}

\nc{\coker}{\on{coker}}
\nc{\End}{\on{End}}
\rnc{\Im}{\on{Im}}
%\rnc{\Re}{\on{Re}}

\nc{\Hom}{\on{Hom}}

\DeclareMathOperator*{\argmin}{arg\,min}
\DeclareMathOperator*{\argmax}{arg\,max}

\usepackage{marginnote}
\nc{\acts}{\curvearrowright}

\nc{\Mat}{\on{Mat}}

\newenvironment{cd}{\begin{equation*}\begin{tikzcd}}{\end{tikzcd}\end{equation*}\ignorespacesafterend}

\nc{\pfrac}[2]{\frac{\partial #1}{\partial #2}}
\nc{\e}[1]{\begin{align*} #1 \end{align*}}

\usepackage[margin=1in]{geometry}

\makeatletter
\def\blfootnote{\gdef\@thefnmark{}\@footnotetext}
\makeatother

%\renewcommand*{\arraystretch}{1.4}

\setlength{\parskip}{0.25cm}

\definecolor{myblue}{rgb}{0,0.15,0.45}
\newenvironment{myproof}{\color{myblue}\begin{proof}}{\end{proof}} 


\usepackage{bm}

\usepackage{fancyhdr}
\pagestyle{fancy} 
\fancyhead[L]{James Tao}
\fancyhead[C]{18.06 -- Week 12 Recitation}
\fancyhead[R]{May 5, 2020}
\fancyfoot[C]{}

\newcounter{part-count}
\setcounter{part-count}{0}

\newenvironment{me}[1]{\begin{enumerate}[#1]\setcounter{enumi}{\value{part-count}}}{\setcounter{part-count}{\value{enumi}}\end{enumerate}}

\nc{\myhead}[1]{\noindent\emph{#1}.}


\begin{document}
	\thispagestyle{fancy}
	
	\section{Statistics and SVD} 
	
	See Section 7.3 of Strang's book. 
	
	Suppose we do $n$ trials of an experiment, and each trial results in a measurement $\bm{v}_i \in \BR^m$. We obtain a data set which is an $n$-tuple of $m$-vectors: $\bm{v}_1, \ldots, \bm{v}_n$. 
	\begin{itemize}
		\item \emph{Example}. On $n$ different days, we measure the temperature and humidity. The measurement for day $i$ is a two-vector $\bm{v}_i = (\text{temp on day } i, \text{humidity on day } i) \in \BR^2$. 
	\end{itemize}
	The \emph{average} measurement is 
	\[
		\bm{\mu} = \frac{\bm{v}_1 + \cdots + \bm{v}_n}{n}. 
	\]
		
	Let $A$ be the $m \times n$ matrix whose $i$-th column is $\bm{v}_i - \bm{\mu}$. (So each row of $A$ sums to zero, by an earlier homework problem.) Then 
	\e{
		S &= \frac{AA^\top}{n-1} \\
		&= \frac{(\bm{v}_1 - \bm{\mu})(\bm{v}_1 - \bm{\mu})^\top + \cdots+ (\bm{v}_n - \bm{\mu})(\bm{v}_n - \bm{\mu})^\top}{n-1}
	}
	is the \emph{sample covariance matrix}.\footnote{We will briefly discuss why the denominator is $n-1$ and not $n$, as would usually be the case for a statistical average.} The entry $S_{j_1j_2}$ is large if the $j_1$ and $j_2$ coordinates tend to have the same sign, and it is negative if these coordinates tend to have the opposite sign. The diagonal entry $S_{jj}$ is a sum of squares, hence positive; it measures how much the $j$-th coordinate tends to vary. 
	\begin{itemize}
		\item In the previous example, $S$ would be a $2 \times 2$ matrix. The entries $S_{11}$ and $S_{22}$ tell you the variance of the temperature and humidity, respectively. The entry $S_{12}$ tells you how temperature correlates with humidity. 
	\end{itemize}
	The `correlation' between two measured variables, such as temperature and humidity, is more frequently expressed by drawing a `line of best fit' on a scatter plot. The SVD of $A$ will tell us how to do this. 
	
	The matrix $S$ is symmetric and positive definite, so it admits an orthonormal basis of eigenvectors, each with positive eigenvalue. Let $\bm{u}_1, \bm{u}_2, \ldots, \in \BR^m$ be the eigenvectors, and let $\sigma_1^2 > \sigma_2^2 > \cdots$ be the corresponding eigenvalues. 
	\begin{itemize}
		\item \emph{Relation with the SVD of $A$}. The vector $\bm{u}_i$ is also a singular vector of $A$ with singular value $(n-1)\sigma_i$. This relationship between the SVD of a matrix $A$ and the eigendecomposition of the matrix $AA^\top$ is used in one of the proofs that an SVD always exists. 
		\item \emph{The statistical meaning}. What do these eigenvectors and eigenvalues tell us? Consider the following `multivariate Gaussian distribution': 
		\begin{enumerate}
			\item On step $i$, choose new values for the scalars $z_1, z_2, \ldots$ based on a bell curve centered at zero with variance 1. 
			\item Set $\bm{v}_i = \bm{\mu} + z_1 \sigma_1 \bm{u}_1 + z_2 \sigma_2 \bm{u}_2 + \cdots$. 
		\end{enumerate}
		The `total variance' is $\sigma_1^2 + \sigma_2^2 + \cdots$, and the `amount of variance explained by $\bm{u}_i$' is $\sigma_i^2$. 
		
		The assertion is that our measurements $\bm{v}_i$ `look like' they are generated in this way. 
	\end{itemize}
	Since $\sigma_1$ is relatively large, $\bm{u}_1$ is the direction in which the data fluctuate away from the mean $\bm{\mu}$ most wildly. The line through $\mu$ parallel to $\bm{u}_1$ is the line of best fit (which minimizes the \emph{perpendicular} distances to the points in the data set). 
	
	We will look at the following example: 
	\[
		A = \begin{pmatrix}
		6 & 5 & -4 & -3 \\
		3 & 4 & -5 & -6
		\end{pmatrix}. 
	\]
	
	\section{Fourier series} 
	
	See Section 10.5 in Strang's book. 
	
	We focus on Example 3 from that section. 
	\begin{itemize}
		\item \emph{Problem}. Express the square wave function 
		\[
			f(x) = \begin{cases}
				1 & \text{ if } \lfloor \tfrac{x}{\pi} \rfloor \text{ is even} \\
				-1 & \text{ otherwise}
			\end{cases}
		\]
		as an infinite linear combination of the functions $\sin(x), \cos(x), \sin(2x), \cos(2x), \ldots$. 
	\end{itemize}
	The crux of the computation is the integral 
	\e{
		\int_{0}^{2\pi} f(x) \, \sin(nx)\, dx  &= \int_{0}^\pi \sin (nx)\, dx - \int_{\pi}^{2\pi} \sin (nx)\, dx \\
		&= \left[ -\tfrac{1}{n} \cos(nx) \right]_{x=0}^\pi - \left[ -\tfrac{1}{n} \cos(nx) \right]_{x=\pi}^{2\pi} \\
		&= \begin{cases}
			\tfrac{4}{n} &\text{ if $n$ is odd} \\
			0 & \text{ otherwise}. 
		\end{cases}
	} 
	We also need the `orthonormality' relations 
	\e{
		\int_0^{2\pi} \sin(nx) \sin(mx)\, dx &= \tfrac12 \left( \int_0^{2\pi} \cos((n-m)x)\, dx - \int_0^{2\pi} \cos((n+m)x)\, dx \right) \\
		&= \begin{cases}
			\pi & \text{ if } n = m \\
			0 & \text{ otherwise}. 
		\end{cases}
	}
	assuming that $n$ and $m$ are positive integers. 
	
	Once we find the answer for $f(x)$, taking the derivatives gives the Taylor series for $f'(x)$, which is an infinite sum of $\delta$-functions. 
	
	Another phenomenon to ponder: (pointwise) multiplication by $\sin(x)$ yields a linear map 
	\[
		(\text{functions with period } 2\pi) \to (\text{functions with period } 2\pi)
	\]
	which is not invertible, but has nullspace $= \{0\}$. Hence, some of the facts we learned in the finite-dimensional setting break down in this infinite-dimensional vector space. 
	
	
	
	
	
	
	
	
	
	
	
	
	
	
	
	
\end{document} 