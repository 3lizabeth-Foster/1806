\documentclass{article}
\usepackage{amsmath, amsthm, amssymb, amsfonts, dsfont, fancyhdr, graphicx, color, tabularx, enumitem}
\usepackage{geometry}


\theoremstyle{definition}
\newtheorem{prob}{}
\renewcommand{\qedsymbol}{}
\renewcommand*{\proofname}{Solution}
\newcommand{\MSB}[1]{\textcolor{blue}{[MSB: #1]}}

\pagestyle{fancy} \fancyhf{} \lhead{\textsc{18.06}} \rhead{10/18/22} 

\begin{document}


\section*{Practice Problems}
\begin{prob}
	When does the least square solution $\hat{x}$ minimizing $||b-Ax||$ give an exact solution to $Ax=b$?
\end{prob}

\begin{proof}
	The vector $\hat{x}$ gives an exact solution to $Ax=b$ exactly if $A\hat{x}=b$. We would like to understand when this happens.
	
	Notice that $A \hat{x}$ is in the column space of $A$, so if $b$ is not in $C(A)$ then $\hat{x}$ can't give an exact solution. On the other hand, if $b$ is in the column space, we claim that $\hat{x}$ is always an exact solution. Indeed, if $b$ is in the column space, then we can find an $x$ so that $Ax=b$. So the minimum value of $||b-Ax||$ is 0. Since $\hat{x}$ minimizes this distance, we must have that $||b-A\hat{x}||=0$. In other words, $A\hat{x}=b$ so $\hat{x}$ is an exact solution as claimed.
\end{proof}

\begin{prob}
	We want to project onto the plane $x-y-2z=0$. To do this, choose 2 vectors spanning that plane (the nullspace of what matrix?) and make them the columns of a matrix $A$ so that $C(A)$ is the plane. Then compute the projection of the point 
	\[b=\begin{pmatrix}
		0\\6\\12
	\end{pmatrix}\]
onto this plane.
\end{prob}
\begin{proof}
	The plane $x-y-2z=0$ is the nullspace of 
	\[B=\begin{pmatrix}
		1 & -1 & -2
	\end{pmatrix}.\]

Since it's a small example, we can eyeball a basis of this plane; the vectors $(1 \ 1 \ 0)$ and $(2\ 0\ 1)$ work. So we have 
\[A=\begin{pmatrix}
	1 & 2\\
	1 & 0\\
	0 & 1
\end{pmatrix}.\]
The projection $p$ of $b$ onto $C(A)$ is 
\[p=A(A^TA)^{-1}A^Tb.\]
To avoid taking inverses, we use the strategy of setting $\hat{x}=(A^TA)^{-1}A^Tb$, solving the normal equations $A^T A \hat{x}= A^T b$, and then computing $p = A \hat{x}$.

We compute: 
\[A^T A= \begin{pmatrix}
	2 & 2\\
	2 & 5
\end{pmatrix} 
\qquad
A^T b= \begin{pmatrix}
	6\\12
\end{pmatrix}.\]
So we have the equation 
\[\begin{pmatrix}
	2 & 2\\
	2 & 5
\end{pmatrix} \begin{pmatrix}
x_1\\x_2
\end{pmatrix}=\begin{pmatrix}
6\\12
\end{pmatrix}.\]
Solving (however you like), we get $x_1=1$ and $x_2=2$. Now, we need to compute $A \hat{x}:$
\[\begin{pmatrix}
	1 & 2\\
	1 & 0\\
	0 & 1
\end{pmatrix} \begin{pmatrix}
1\\2
\end{pmatrix}=\begin{pmatrix}
5 \\ 1\\2
\end{pmatrix}= p.\]

Alternate solution: we know that $b=p + e$, where $p$ is the projection of $b$ to the plane $x-y-2z=0$ and $e$ is the projection of $b$ to the orthogonal complement of that plane. So if we can compute $e$, we can compute $p=b-e$. 

The orthogonal complement of the plane is the vector space $S$ spanned by $v=(1 \ -1\ -2)$. Using the formula from lecture, the projection of $b$ onto $S$ is
\[e=\frac{v v^T}{v^T v}b.\]
Now, $v^Tv=6$ and 
\[vv^T=\begin{pmatrix}	1\\-1\\-2 \end{pmatrix}\begin{pmatrix}	1&-1&-2 \end{pmatrix}=\begin{pmatrix}
	1&-1&-2\\-1&1&2\\-2&2&4
\end{pmatrix}
\]
so 
\[e=\frac{1}{6}\begin{pmatrix}
	1&-1&-2\\-1&1&2\\-2&2&4
\end{pmatrix}\begin{pmatrix}
0\\6\\12
\end{pmatrix}=\begin{pmatrix}
-5\\5\\10
\end{pmatrix} \]
and 
\[p=b-e=\begin{pmatrix}
	0\\6\\12
\end{pmatrix}-\begin{pmatrix}
	-5\\5\\10
\end{pmatrix}=\begin{pmatrix}
5\\1\\2
\end{pmatrix}\]
\end{proof}

\begin{prob}
	Say $P$ is an $m \times m$ orthogonal-projection matrix onto an $n$-dimensional subspace. What is the rank of $A=(I-P)P$? What is the rank of $B=(I-P)-P$? Show that $B$ is orthogonal. (Hint: helpful to compute $B^2$, and also to draw a picture of what $B$ does.)
\end{prob}
\begin{proof}
	Let's say $S$ is the subspace $P$ projects onto.
	
	For the first question, we multiply 
	\[A=(I-P)P=P-P^2=P-P=0.\]
So $A$ has rank 0. This makes sense; $I-P$ is the projection onto $S^\perp$. So $A$ first projects $x$ to $S$, and then projects the resulting point of $S$ to $S^\perp$. But if $s$ is a vector in $S$, the closest point in $S^\perp$ will always be 0.
	
	For the second question, it's easiest to picture the case where we're in the plane and $S$ is $1$-dimensional (so a line through the origin). In that case, $B$ takes a vector and reflects it across the line through the origin perpendicular to $S$; that is $B$ is reflection across $S^\perp$. This gives us the intuition that $B$ is full rank, since reflection is an invertible transformation. We'll show this using the hint.
	
	Taking the hint, we compute 
	\[B^2=I -4P + 4P^2=I-4P+4P=I\]
	where we are using the fact that $P^2=P$. This means that $B$ is invertible (in fact, $B$ is its own inverse) and in particular is rank $m$.
	
	We've already computed that $B^2=I$, so $B^{-1}=B$. We'd like to show that $B$ is \emph{orthogonal}; that is, that $B^{-1}=B^T$. It's enough to show that $B=B^{T}$. Remember from class that for a projection matrix $P$, $P=P^T$. So we check
	\[B^T=(I-2P)^T=I-2P^T=I-2P=B\]
	as desired.
\end{proof}

\begin{prob}
	If $A$ is $m \times n$ with rank $n$, what is the complexity of finding the projection $p$ onto $C(A)$ of a point $b$ by 
	\begin{itemize}
		\item[(1)] forming the projection matrix $P$ (using the formula from class) then doing the multiplication $Pb$
		\item[(2)] forming the normal equations, solving them for $\hat{x}$ and then computing $p=A \hat{x}$.
	\end{itemize}
\end{prob}

\begin{proof}
For (1): The formula is 
\[P=(A^TA)^{-1}A^T.\]
To compute $A^T$, you do roughly $mn$ arithmetic operations (there are $mn$ entries of $A^T$ and to get a single entry, you do 1 operation, which is selecting a particular entry of $A$). Computing the product $A^T A$ uses $mn^2$ operations (there are $n^2$ entries of $A^T A$ and to get a single entry, you do about $m$ operations). Inverting $A^T A$ uses $n^3$ operations (e.g. using Gaussian elimination). Doing the final multiplication uses $m n^2$ operations (there are $mn$ entries of $P$ and computing one of them takes about $n$ operations). Since $m>n$, the overall computational complexity is on the order of $mn^2$.

For (2): Notice that the most expensive part of the above computation was actually multiplying the matrices $A^TA$, which took $mn^2$ operations. This will also be the most expensive part of this method. The normal equations are $A^T A \hat{x}= A^T b$. Computing $A^T b$ takes $mn$ operations; solving the equations takes $n^3$ operations; multiplying $A\hat{x}$ in the end takes $mn$ operations. So again, the overall computational complexity is on the order of $mn^2$.
\end{proof}


 \end{document}