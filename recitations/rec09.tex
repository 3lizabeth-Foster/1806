\documentclass{article}
\usepackage{amsmath, amsthm, amssymb, amsfonts, dsfont, fancyhdr, graphicx, color, tabularx, enumitem}
\usepackage{geometry}


\theoremstyle{definition}
\newtheorem{prob}{}
\renewcommand{\qedsymbol}{}
\renewcommand*{\proofname}{Solution}
\newcommand{\MSB}[1]{\textcolor{blue}{[MSB: #1]}}
\def\tr{\text{tr}}

\pagestyle{fancy} \fancyhf{} \lhead{\textsc{18.06}} \rhead{11/15/22} 

\begin{document}


\section*{Practice Problems}
\begin{prob}
	Consider the matrix
	\[A= \begin{pmatrix}
		1 & -1 & -1\\-1&1&-1\\-1&-1&1
	\end{pmatrix}.\]
One eigenvalue of $A$ is $-1$. The other eigenvalue is a double root of $\det(A-\lambda I)$. What is the other eigenvalue? (Hint: there is a way to do this without computing the characteristic polynomial. What are some equations you know involving eigenvalues?)
\end{prob}


\begin{prob}
Suppose $M$ is a positive Markov matrix (so one eigenvalue equals $1$, all other eigenvalues have $|\lambda|<1$).  Why is $M^\infty$ a rank-1 matrix?
\end{prob}


\begin{prob}
	$x^T A y = \text{tr}(A B)$ where $B$ is what matrix?   (Hint: recall that the trace of a 1x1 matrix $a$ is $a$).
\end{prob}


\begin{prob}

Suppose $A$ is an $m \times n$ full column-rank matrix with thin SVD $U\Sigma V^T$.   By inspection of $A^T A$ in comparison with the diagonalization formula, the eigenvectors of $A^T A$ are \_\_\_ and its eigenvalues are \_\_\_ .
\end{prob}



\begin{prob}
	Suppose $A$ is $m \times m$, full rank, and we compute its QR factorization $A=QR$, e.g. by Gram–Schmidt.   Claim: the matrix $B=RQ$ has the same eigenvalues as $A$.  Why?
\end{prob}




\end{document}