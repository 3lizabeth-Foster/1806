\documentclass[10pt]{amsart} 


\usepackage{amsmath, amssymb, mathrsfs} 

\usepackage[mathscr]{euscript} 
 
\newlength{\mylength}
\setlength{\mylength}{0.25cm}

\usepackage{enumitem}
\setlist{listparindent=\parindent, itemsep=0cm, parsep=\mylength, topsep=0cm}

\usepackage[final]{todonotes}
\usepackage[final]{showkeys} 

\usepackage[breaklinks=true]{hyperref} 
\usepackage{comment} 

\usepackage{url}

\usepackage{tikz-cd}

\usepackage{amsthm}

\makeatletter
\renewenvironment{proof}[1][\proofname]{\par
	\pushQED{\qed}%
	\normalfont \topsep6\p@\@plus6\p@\relax
	\noindent\emph{#1.} 
	\ignorespaces
}{%
\popQED\endtrivlist\@endpefalse
}
\makeatother

\newtheoremstyle{mythm}% name of the style to be used
{\mylength}% measure of space to leave above the theorem. E.g.: 3pt
{0pt}% measure of space to leave below the theorem. E.g.: 3pt
{\itshape}% name of font to use in the body of the theorem
{0pt}% measure of space to indent
{\bfseries}% name of head font
{.\ }% punctuation between head and body
{ }% space after theorem head; " " = normal interword space
{\thmname{#1}\thmnumber{ #2}\thmnote{ (#3)}}

\newtheoremstyle{myrmk}% name of the style to be used
{\mylength}% measure of space to leave above the theorem. E.g.: 3pt
{0pt}% measure of space to leave below the theorem. E.g.: 3pt
{}% name of font to use in the body of the theorem
{0pt}% measure of space to indent
{\itshape}% name of head font
{.\ }% punctuation between head and body
{ }% space after theorem head; " " = normal interword space
{\thmname{#1}\thmnumber{ #2}\thmnote{ (#3)}}

\theoremstyle{mythm} 
%\newtheorem{thm}[subsubsection]{Theorem}
%\newtheorem*{claim}{Claim}
%\newtheorem*{thm}{Theorem} 
\newtheorem{thm}{Theorem}
\newtheorem{lem}[thm]{Lemma} 
\newtheorem{cor}[thm]{Corollary}
\newtheorem{claim}[thm]{Claim}
\newtheorem{prop}[thm]{Proposition}
%\newtheorem*{mthm}{Main Theorem}

%\newtheorem{prop}[subsubsection]{Proposition} 
%\newtheorem*{prop}{Proposition} 
%\newtheorem*{lem}{Lemma}
%\newtheorem*{klem}{Key Lemma}
%\newtheorem*{cor}{Corollary}

\theoremstyle{definition}
%\newtheorem{defn}[subsubsection]{Definition}
\newtheorem*{defn}{Definition} 
\newtheorem{prob}[thm]{Problem}
%\newtheorem{que}[subsubsection]{Question}

\theoremstyle{myrmk} 
%\newtheorem{rmk}[subsubsection]{Remark}
\newtheorem*{rmk}{Remark}
%\newtheorem{note}[subsubsection]{Note} 
\newtheorem*{ex}{Example}

\newcommand{\nc}{\newcommand} 
\nc{\on}{\operatorname}
\nc{\rnc}{\renewcommand} 

\rnc{\setminus}{\smallsetminus} 

\nc{\wt}{\widetilde}
\nc{\wh}{\widehat} 
\nc{\ol}{\overline} 

\nc{\Frob}{\on{Frob}}
\nc{\Gal}{\on{Gal}}

\nc{\BN}{\mathbb{N}}
\nc{\BZ}{\mathbb{Z}}
\nc{\BQ}{\mathbb{Q}}
\nc{\BR}{\mathbb{R}}
\nc{\BC}{\mathbb{C}}

\nc{\id}{\on{id}}
\nc{\Id}{\on{Id}}
\nc{\Tr}{\on{Tr}}

\nc{\la}{\langle}
\nc{\ra}{\rangle} 
\nc{\lV}{\lVert}
\nc{\rV}{\rVert}
\nc{\mb}{\mathbf}
\nc{\mf}{\mathfrak}
%\nc{\cur}{\mathscr}
\nc{\mc}{\mathscr}

\nc{\ira}{\hookrightarrow}
\nc{\hra}{\hookrightarrow}
\nc{\sra}{\twoheadrightarrow} 

\rnc{\Re}{\on{Re}}

\nc{\coker}{\on{coker}}
\nc{\End}{\on{End}}
\rnc{\Im}{\on{Im}}
%\rnc{\Re}{\on{Re}}

\nc{\Hom}{\on{Hom}}

\DeclareMathOperator*{\argmin}{arg\,min}
\DeclareMathOperator*{\argmax}{arg\,max}

\usepackage{marginnote}
\nc{\acts}{\curvearrowright}

\nc{\Mat}{\on{Mat}}

\newenvironment{cd}{\begin{equation*}\begin{tikzcd}}{\end{tikzcd}\end{equation*}\ignorespacesafterend}

\nc{\pfrac}[2]{\frac{\partial #1}{\partial #2}}
\nc{\e}[1]{\begin{align*} #1 \end{align*}}

\usepackage[margin=1in]{geometry}

\makeatletter
\def\blfootnote{\gdef\@thefnmark{}\@footnotetext}
\makeatother

%\renewcommand*{\arraystretch}{1.4}

\setlength{\parskip}{0.25cm}

\title{Spring 18.06 - Week 5 Recitation} 
\author{James Tao}


\usepackage{fancyhdr}
\pagestyle{fancy} 
\fancyhead[L]{James Tao}
\fancyhead[C]{Spring 18.06 -- Week 5 Recitation}
\fancyhead[R]{Mar.\ 2, 2020}
\fancyfoot[C]{}


\begin{document}
	\thispagestyle{fancy}
	
	Topic: four subspaces associated to a matrix. 
	
	\begin{enumerate}[label=(\arabic*), itemsep = 0.6in]
		\item Let $A$ be an $n \times n$ invertible matrix. Then $\on{col}(A) = \BR^n$ and $\on{null}(A) = \{0\}$. 
		\item Let $A$ and $B$ be matrices such that $A$ is invertible and $AB$ makes sense. Then 
		\e{
			\on{null}(AB) &= \on{null}(B) \\
			\on{col}(AB) &= \{Ax \text{ for } x \in \on{col}(B)\} \\
			\on{rank}(AB) &= \on{rank}(B). 
		} 
		\item Let $A$ and $B$ be matrices such that $A$ is invertible and $BA$ makes sense. Then 
		\e{
			\on{col}(BA) &= \on{col}(B) \\
			\on{null}(BA) &= \{A^{-1}x \text{ for } x \in \on{null}(B)\} \\
			\on{rank}(BA) &= \on{rank}(B). 
		} 
		\item Let $A$ be an $n \times n$ invertible matrix. Let $r \le n$, and let $B$ be the $n \times r$ matrix built from the first $r$ columns of $A$. Then 
		\[
			B = A\left( \begin{array}{c} \on{Id}_{r \times r} \\ \hline 0_{(n-r) \times r}\end{array} \right). 
		\]
		Use (2) to deduce $\on{null}(B) = 0$ and $\on{rank}(B) = r$. 
		\item Let $A$ be an $n \times n$ invertible matrix. Let $r \le n$, and let $B$ be the $r \times n$ matrix built from the first $r$ rows of $A$. Then 
		\[
			B = \left( \begin{array}{c|c} \on{Id}_{r \times r} & 0_{(n-r) \times r}\end{array} \right) A. 
		\]
		Use (3) to deduce $\on{col}(B) = \BR^n$ and $\on{null}(B)$ is the span of the last $(n-r)$ columns of $A^{-1}$. 
		\item Let $A$ be an $n \times m$ matrix, and fix an integer $r \le n, m$.  Assume that $A = EFG$ where $E$ is an $n \times r$ matrix arising from the construction in (4), $F$ is an invertible $r \times r$ matrix, and $G$ is an $r \times m$ matrix arising from the construction in (5). Then 
		\e{
			\on{col}(A) &= \on{col}(E) \\
			\on{rank}(A) &= r \\
			\on{null}(A) &= \on{null}(G). 
		}
		\item Suppose $P$ is a square matrix such that $P^2 = P$. Then $b \in \on{col}(P)$ if and only if $Pb = b$. 
		\item Let $Q$ be an orthogonal matrix. Then $b \in \on{col}(Q)$ if and only if $QQ^\top b = b$. 
		\item Let $A = QR$ be a $QR$ decomposition (where $R$ is invertible). Then $b \in \on{col}(A)$ if and only if $Q Q^\top b = b$. 
		\item\footnote{Thanks to Sungwoo for this problem idea.} Let $A = U\Sigma V^\top$ be a rank-$r$ SVD. Then $b \in \on{col}(A)$ if and only if $UU^\top b = b$. 
		\item Determine the column space, null space, and rank of the matrix 
		\[
			A = \begin{pmatrix}
			1 & 0 & 0 \\ 2 & 1 & 0 \\ 5 & 3 & 1
			\end{pmatrix}
			\begin{pmatrix}
			1 & 4 & 2 & 3 \\ 0 & 0 & 1 & 5 \\ 0 & 0 & 0 & 0
			\end{pmatrix}. 
		\]
		\item Determine the column space, null space, and rank of the matrix 
		\[
			A = \begin{pmatrix}
			\frac{1}{\sqrt2} & \frac{1}{\sqrt2}  & 0 \\ 
			\frac{1}{\sqrt2} & -\frac{1}{\sqrt2}  & 0 \\
			0 & 0 & 0 \\
			0 & 0 & 1 
			\end{pmatrix} \begin{pmatrix}
			1 & 2 & 3 \\ 
			0 & 6 & 8 \\
			0 & 0 & 7
			\end{pmatrix}. 
		\]
		
		\item Consider the following full SVD of a matrix:  
		\[
			A = \begin{pmatrix}
			\frac{1}{\sqrt2} & \frac{1}{\sqrt2}  & 0 \\ 
			\frac{1}{\sqrt2} & -\frac{1}{\sqrt2}  & 0 \\
			0 & 0 & 1
			\end{pmatrix}
			\begin{pmatrix}
				6 & 0 & 0 \\
				0 & 1 & 0 \\
				0 & 0 & 0 
			\end{pmatrix}
			\begin{pmatrix}
				0 & 0 & 1 \\
				0 & 1 & 0 \\
				1 & 0 & 0 
			\end{pmatrix}
		\]
		Write down the rank-$r$ SVD for this matrix, and determine its column space, null space, and rank.
		\item Any $n \times m$ matrix can be expressed as the sum of (at most) $\on{min}(m, n)$ rank-1 matrices. 
	\end{enumerate}
	
\end{document} 